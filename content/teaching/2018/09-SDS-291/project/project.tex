\documentclass[]{article}
\usepackage{lmodern}
\usepackage{amssymb,amsmath}
\usepackage{ifxetex,ifluatex}
\usepackage{fixltx2e} % provides \textsubscript
\ifnum 0\ifxetex 1\fi\ifluatex 1\fi=0 % if pdftex
  \usepackage[T1]{fontenc}
  \usepackage[utf8]{inputenc}
\else % if luatex or xelatex
  \ifxetex
    \usepackage{mathspec}
    \usepackage{xltxtra,xunicode}
  \else
    \usepackage{fontspec}
  \fi
  \defaultfontfeatures{Mapping=tex-text,Scale=MatchLowercase}
  \newcommand{\euro}{€}
\fi
% use upquote if available, for straight quotes in verbatim environments
\IfFileExists{upquote.sty}{\usepackage{upquote}}{}
% use microtype if available
\IfFileExists{microtype.sty}{%
\usepackage{microtype}
\UseMicrotypeSet[protrusion]{basicmath} % disable protrusion for tt fonts
}{}
\usepackage[margin=1in]{geometry}
\ifxetex
  \usepackage[setpagesize=false, % page size defined by xetex
              unicode=false, % unicode breaks when used with xetex
              xetex]{hyperref}
\else
  \usepackage[unicode=true]{hyperref}
\fi
\hypersetup{breaklinks=true,
            bookmarks=true,
            pdfauthor={},
            pdftitle={Project Instructions},
            colorlinks=true,
            citecolor=blue,
            urlcolor=blue,
            linkcolor=magenta,
            pdfborder={0 0 0}}
\urlstyle{same}  % don't use monospace font for urls
\usepackage{graphicx,grffile}
\makeatletter
\def\maxwidth{\ifdim\Gin@nat@width>\linewidth\linewidth\else\Gin@nat@width\fi}
\def\maxheight{\ifdim\Gin@nat@height>\textheight\textheight\else\Gin@nat@height\fi}
\makeatother
% Scale images if necessary, so that they will not overflow the page
% margins by default, and it is still possible to overwrite the defaults
% using explicit options in \includegraphics[width, height, ...]{}
\setkeys{Gin}{width=\maxwidth,height=\maxheight,keepaspectratio}
\setlength{\parindent}{0pt}
\setlength{\parskip}{6pt plus 2pt minus 1pt}
\setlength{\emergencystretch}{3em}  % prevent overfull lines
\providecommand{\tightlist}{%
  \setlength{\itemsep}{0pt}\setlength{\parskip}{0pt}}
\setcounter{secnumdepth}{0}

%%% Use protect on footnotes to avoid problems with footnotes in titles
\let\rmarkdownfootnote\footnote%
\def\footnote{\protect\rmarkdownfootnote}

%%% Change title format to be more compact
\usepackage{titling}

% Create subtitle command for use in maketitle
\newcommand{\subtitle}[1]{
  \posttitle{
    \begin{center}\large#1\end{center}
    }
}

\setlength{\droptitle}{-2em}

  \title{Project Instructions}
    \pretitle{\vspace{\droptitle}\centering\huge}
  \posttitle{\par}
    \author{}
    \preauthor{}\postauthor{}
    \date{}
    \predate{}\postdate{}
  
% Redefines (sub)paragraphs to behave more like sections
\ifx\paragraph\undefined\else
\let\oldparagraph\paragraph
\renewcommand{\paragraph}[1]{\oldparagraph{#1}\mbox{}}
\fi
\ifx\subparagraph\undefined\else
\let\oldsubparagraph\subparagraph
\renewcommand{\subparagraph}[1]{\oldsubparagraph{#1}\mbox{}}
\fi


\begin{document}
\maketitle

You will conduct a statistical study on a topic of your choice. This
task will require you to write a project proposal, acquire and analyze
relevant data, present your results orally to the class, and hand in a
written report describing your study and its findings. Your project must
involve fitting a \emph{regression} model. The project is an opportunity
to show off what you've learned about data analysis, visualization, and
statistical inference. It is a major component of the class, and
successful completion is required to pass.

\paragraph*{Group Formation}\label{group-formation}
\addcontentsline{toc}{paragraph}{Group Formation}

You will work in a group of three, of your choosing. Apart from unusual
circumstances, all three of you must be in the same section of MTH 291.
Each group will be assigned a letter, and you will use \texttt{Group-X}
with your particular letter in all communications from that point
forward.

\paragraph*{Assignment}\label{assignment}
\addcontentsline{toc}{paragraph}{Assignment}

You should pose a problem that you find interesting and which may be
addressed (at least in part) through the analysis of data. Many
interesting quantitative questions (and perhaps even more uninteresting
ones) involve the relationships among several variables. Recent projects
have considered the following questions:

\begin{itemize}
\item
  What is the role of executive compensation in determining company
  performance?
\item
  What is the relationship between the value of a share of stock and
  financial characteristics of a company?
\item
  How is the state's murder rate affected by its demographics and social
  characteristics?
\item
  How is the percentage of Massachusetts high school seniors going on to
  four-year colleges influenced by town and school characteristics?
\item
  What factors influence the incidence of tuberculosis in the U.S.?
\item
  How can we predict real estate prices in Northampton?
\end{itemize}

You should pose the problem that you want to solve as precisely as
possible at the outset. Next, identify the population you want to
describe, and think about how you will obtain relevant data. What kind
of model might be appropriate in this context? You should also make a
hypothesis, \emph{a priori} (before you analyze the data), about the
results you expect to see.

Most of you will pose your own question and acquire data from the
Internet, some may wish to analyze data that someone else (e.g., a
professor or office at Smith, published data in a magazine or newspaper)
has collected for another purpose. Please consult us if you plan to do
this.

\paragraph{General Rules}\label{general-rules}

You all \emph{must} speak during the oral presentation. You may discuss
your project with other students, but each of you will have a different
topic, so there is a limit to how much you can help each other. You may
consult other sources for information about the non-statistical,
substantive issues in your problem, but you should credit these sources
in your report. Feel free to consult with us about statistical
questions.

\paragraph*{Submission}\label{submission}
\addcontentsline{toc}{paragraph}{Submission}

All deliverables described above must be delivered electronically via
Moodle by 11:55pm (five minutes before midnight) on the dates in the
Syllabus. Only one person from the group should submit the group's
product for each checkpoint (with the exception of the Group Dynamic,
which is individual).

\paragraph{Components}\label{components}

\subparagraph{Roster}\label{roster}

Form a group of 3 students from your section. Have one person send a
group roster electronically by the date listed above, with appropriate
cc's, using the message subject header \texttt{MTH\ 291\ Group\ Roster}.
Take the initiative to ask around and find a group to work with.

\subparagraph{Proposals}\label{proposals}

Count on brainstorming at least half a dozen serious ideas before you
can groom one of them into a mature proposal.

For the most part, the choice of topic is left up to you. Try to pick
something that's interesting yet substantial and worth studying, and aim
for a topic that you think nobody has tried before; remember that part
of your overall grade will be based on originality.

Places to Find Data

Finding the right data to answer your particular question is part of
your responsibility for this assignment. Public data sets are available
from hundreds of different websites, on virtually any topic. You might
not be able to find the exact data that you want, but you should be able
to find data that is relevant to your topic. You may also want to refine
your research question so that it can be more clearly addressed by the
data that you found. But be creative! Go find the data that you want!

Below is a list of places to get started, but this list should be
considered grossly non-exhaustive:

\begin{itemize}
\item
  Finding Data on the Internet
  (\url{http://www.inside-r.org/howto/finding-data-internet})
\item
  Gapminder (\url{www.gapminder.org})
\item
  Data.gov (\url{explore.data.gov})
\item
  StatLib at Carnegie Mellon (\url{http://lib.stat.cmu.edu/})
\item
  U.S. Bureau of Labor Statistics (\url{www.bls.gov})
\item
  U.S. Census Bureau (\url{www.census.gov})
\end{itemize}

Keep the following in mind as you select your topic and dataset:

\begin{itemize}
\item
  You need to have enough data to make meaningful inferences. There is
  no magic number of individuals required for all projects. But aim for
  at least 200 individuals and make sure there are at least 20
  individuals in each category of each of your categorical variables (if
  you have any).
\item
  Most projects will measure a quantitative outcome, with at least two
  other variables included in the dataset (ideally at least one of which
  is quantitative). Most of you will use multiple linear regression for
  your primary analyses.
\end{itemize}

Once we respond to your initial proposal, you will revise it (perhaps
starting with a different dataset), then submit a new proposal that
addresses our feedback. Supply essentially the same information required
for the initial proposal, but give a bit more detail.

\subparagraph{Content}\label{content}

Your initial and revised proposals should contain the following content:

\begin{enumerate}
\def\labelenumi{\arabic{enumi}.}
\item
  Group Members: List the members of your group
\item
  Title: Your title
\item
  Purpose: Describe the general topic/phenomenon you want to study, as
  well some focused questions that you hope to answer and specific
  hypotheses that you intend to assess.
\item
  Data: Describe the data that you plan to use, with specifications of
  where it can be found (URL) and a short description. Eventually, you
  will probably want to combine data from multiple sources into one
  file. We will discuss data management techniques in the coming weeks,
  but for now you should simply list multiple sources if you have them.
\item
  Population: Specify what the observational units are (i.e.~the rows of
  the data frame), describe the larger population/phenomenon to which
  you'll try to generalize, and (if appropriate) estimate roughly how
  many such individuals there are in the population.
\item
  Response Variable: What the response variable? What are its units?
  Estimate the range of possible values that it may take on.
\item
  Explanatory Variables: Describe the variables that you'll examine for
  each observational unit (i.e.~the columns of the data frame).
  Carefully define each variable and describe how each was measured. For
  categorical variables, list the possible categories; for quantitative
  variables, specify the units of measurement. You may want to add more
  variables later on, but you should have at least 5 variables already.
\end{enumerate}

\subparagraph{Data}\label{data}

You must finalize and submit your data file to us. Your data file should
also be placed in your Dropbox folder. Your technical report should
import this data into RStudio using the \texttt{read.csv()} command.

\begin{itemize}
\item
  The data must be in
  \href{http://en.wikipedia.org/wiki/Comma-separated_values}{CSV format}
  (\texttt{.csv}). This means that the first row should be a
  comma-separated list of variable names, and the rest should be rows of
  data.
\item
  Your data file should be named \texttt{group-X-data.csv}.
\item
  Name all variables helpfully and contextually, e.g., use
  \texttt{Airport} and \texttt{WaterTemp}, not \texttt{Individuals} and
  \texttt{Treatments}, and certainly not \texttt{A} and \texttt{B}.
  Similarly, for the category names, use whole words and phrases, not
  cryptic codes, e.g., use \texttt{Male} and \texttt{Female}, not
  \texttt{1} and \texttt{2}. A binary variable \texttt{isFemale} can be
  coded 0 for male, and 1 for female (and then is self-documenting). A
  variable \texttt{sex} coded 1 and 2 is just asking for trouble.
\item
  That said, try to limit your variable and category names to about a
  dozen characters. This may take some abbreviation.
\item
  Check for data integrity! Manual inspection is OK, but it's tedious
  and it's easy to overlook misspellings. Running some simple analyses
  can more quickly make most data entry errors obvious.
\end{itemize}

\subparagraph{Technical Report}\label{technical-report}

Your technical report will be an annotated R Markdown file
(\texttt{.Rmd}) that contains your R code, interspersed with
explanations of what the code is doing, and what it tells you about the
problem.

Content

You should \textbf{not} need to present \emph{all} of the R code that
you wrote throughout the process of working on this project. Rather, the
technical report should contain the \emph{minimal} set of R code that is
necessary to understand your results and findings in full. If you make a
claim, it \emph{must} be justified by explicit calculation. A
knowledgeable reviewer should be able to compile your \texttt{.Rmd} file
without modification, and verify every statement that you have made. All
of the R code necessary to produce your figures and tables \emph{must}
appear in the technical report. In short, the technical report should
enable a reviewer to reproduce your work in full.

Tone

This document should be written for peer reviewers, who comprehend
statistics at least as well as you do. You should aim for a level of
complexity that is more statistically sophisticated than an article in
the \href{http://www.nytimes.com/pages/science/}{Science section of
\emph{The New York Times}}, but less sophisticated than an academic
journal. {[}\href{http://chance.amstat.org/}{\emph{Chance} magazine}
might provide a good example.{]} For example, you may use terms that
that you will likely never see in the \emph{Times} (e.g.~bootstrap), but
should not dwell on technical points with no obvious ramifications for
the reader (e.g.~reporting F-statistics). Your goal for this paper is to
convince a statistically-minded reader (e.g.~a student in this class, or
a student from another school who has taken an introductory statistics
class) that you have addressed an interesting research question in a
meaningful way. Even a reader with no background in statistics should be
able to read your paper and get the gist of it.

Format

Your technical report should follow this basic format:

\begin{enumerate}
\def\labelenumi{\arabic{enumi}.}
\item
  Abstract: a short, one paragraph explanation of your project. The
  abstract should not consist of more than 5 or 6 sentences, but should
  relate what you studied and what you found. It need only convey a
  general sense of what you actually did. The purpose of the abstract is
  to give a prospective reader enough information to decide if they want
  to read the full paper.
\item
  Introduction: an overview of your project. In a few paragraphs, you
  should explain \emph{clearly} and \emph{precisely} what your research
  question is, why it is interesting, and what contribution you have
  made towards answering that question. You should give an overview of
  the specifics of your model, but not the full details. Most readers
  never make it past the introduction, so this is your chance to hook
  the reader, and is in many ways the most important part of the paper!
\item
  Data: a brief description of your data set. What variables are
  included? Where did they come from? What are units of measurement?
  What is the population that was sampled? How was the sample collected?
  You should also include some basic univariate analysis.
\item
  Results: an explanation of what your model tells us about the research
  question. You should interpret coefficients in context and explain
  their relevance. What does your model tell us that we didn't already
  know before? You may want to include negative results, but be careful
  about how you interpret them. For example, you may want to say
  something along the lines of: ``we found no evidence that explanatory
  variable \(x\) is associated with response variable \(y\),'' or
  ``explanatory variable \(x\) did not provide any additional
  explanatory power above what was already conveyed by explanatory
  variable \(z\).'' On other hand, you probably shouldn't claim: ``there
  is no relationship between \(x\) and \(y\).''
\item
  Conclusion: a summary of your findings and a discussion of their
  limitations. First, remind the reader of the question that you
  originally set out to answer, and summarize your findings. Second,
  discuss the limitations of your model, and what could be done to
  improve it. You might also want to do the same for your data. This is
  your last opportunity to clarify the scope of your findings before a
  journalist misinterprets them and makes wild extrapolations! Protect
  yourself by being clear about what is \emph{not} implied by your
  research.
\end{enumerate}

Additional Thoughts

The technical report is \emph{not} simply a dump of all the R code you
wrote during this project. Rather, it is a narrative, with technical
details, that describes how you addressed your research question. You
should \emph{not} present tables or figures without a written
explanation of the information that is supposed to be conveyed by that
table or figure. Keep in mind the distinction between \emph{data} and
\emph{information}. Data is just numbers, whereas information is the
result of analyzing that data and digesting it into meaningful ideas
that human beings can understand. Your technical report should allow a
reviewer to follow your steps from converting data into information.
There is no limit to the length of the technical report, but it should
not be longer than it needs to be. You will not receive extra credit for
simply describing your data \emph{ad infinitum}. For example, simply
displaying a table with the means and standard deviations of your
variables is not meaningful. Writing a sentence that reiterates the
content of the table (e.g. ``the mean of variable \(x\) was 34.5 and the
standard deviation was 2.8\ldots{}'') is equally meaningless. What you
should strive to do is interpret these values in context (e.g.
``although variables \(x_1\) and \(x_2\) have similar means, the spread
of \(x_1\) is much larger, suggesting\ldots{}'').

You should present figures and tables in your technical report in
context. These items should be understandable on their own -- in the
sense that they have understandable titles, axis labels, legends, and
captions. Someone glancing through your technical report should be able
to make sense of your figures and tables without having to read the
entire report. That said, you should also include a discussion of what
you want the reader to learn from your figures and tables.

Your report should be submitted via email as an R Markdown
(\texttt{.Rmd}) file and the corresponding rendered output
(\texttt{.html}) file.

\subparagraph{Presentation}\label{presentation}

An effective oral presentation is an integral part of this project. One
of the objectives of this class is to give you experience conveying the
results of a technical investigation to a non-technical audience in a
way that they can understand. Whether you choose to stay in academia or
pursue a career in industry, the ability to communicate clearly is of
paramount importance. As a data analyst, the burden of proof is on you
to convince your audience that what you are saying is true. If your
audience (who may very well be less knowledgeable about statistics than
you are) cannot understand your results or their interpretations, then
the technical merit of your project is irrelevant.

You will make a 10-minute oral presentation to the class. You should
make (good) slides. Your goal should be to convey to your audience a
clear understanding of your research question, along with a basic
understanding of your model, and how well it addresses the research
question you posed. You should \textbf{not} tell us everything that you
did, or show a bunch of models that didn't work well. After hearing your
talk, each student in the class should be able to answer:

\begin{enumerate}
\def\labelenumi{\arabic{enumi}.}
\item
  What was your project about?
\item
  What kind of model did you build?
\item
  How well did it work?
\end{enumerate}

You should prepare electronic slides for your talk. PowerPoint is fine,
but you might also want to consider Google Presentation, Beamer (LaTeX),
or alternative, non-linear presentation software like Prezi. Use your
creativity! One thing you should \emph{not} do is walk us through your
calculations in RStudio. There will be an opportunity to rehearse your
presentation with one of us a few days before your talk.

You will need to submit your slides before your presentation, but you
should also bring the slides on a flash drive as a backup. You will not
be able to hook up your laptop to the computer in SR 301.

\subparagraph{Advice}\label{advice}

There are many sources of advice for how to make a good presentation,
but an excellent place to start is
\href{http://techspeaking.denison.edu/}{Technically Speaking}. Watch the
videos on this site to identify some common mistakes. You should also
read Joe Gallian's article on how to make a good presentation.

Here are is some general advice:

\begin{itemize}
\item
  Budget your time. You only have 10 minutes, and we will be running a
  very tight schedule. If your talk runs too short or too long, it makes
  you seem unprepared. Rehearse your talk ahead of time (with your
  group) several times in order to get a better feel for your timing.
  Note also that you may have a tendency to talk faster during your
  actual talk than you will during your rehearsal. Talking faster in
  order to speed up is not a good strategy -- you are much better off
  simply cutting material ahead of time. You will probably have a hard
  time getting through 10 slides in 10 minutes.
\item
  Don't write too much on each slide. You don't want people to have to
  read your slides, because if the audience is reading your slides, then
  they aren't listening to you. You want your slides to provide visual
  cues to the points that you are making -- not substitute for your
  spoken words. Concentrate on graphical displays and bullet-pointed
  lists of ideas.
\item
  Put your problem in context. Remember that most of your audience will
  have little or no knowledge of your subject matter. The easiest way to
  lose people is to dive right into technical details that require prior
  `domain knowledge." Spend a few minutes at the beginning of your talk
  introducing your audience to the most basic aspects of your topic and
  present some motivation for what you are studying.
\item
  Speak loudly and clearly. Remember that you know more about your topic
  that anyone else in the room, so speak and act with confidence!
\item
  Tell a story -- not necessarily the whole story. It is unrealistic to
  expect that you can tell your audience everything that you know about
  your topic in 10 minutes. You should strive to convey the big ideas in
  a clear fashion, but not dwell on the details. Your talk will be
  successful if your audience is able to walk away with an understanding
  of what your research question was, how you addressed it, and what the
  implications of your findings are.
\end{itemize}

\subparagraph{Group Dynamic Report}\label{group-dynamic-report}

Ideally, all group members would be equally involved and able and
committed to the project. In reality, it doesn't always work that way.
We'd like to reward people fairly for their efforts in this group
endeavor, because it's inevitable that there will be variation in how
high a priority people put on this class and how much effort they put
into this project.

To this end, we'd like each of you (individually) to describe how well
(or how poorly!) your project group worked together and shared the load.
Also give some specific comments describing each member's overall
effort. Were there certain group members who really put out exceptional
effort and deserve special recognition? Conversely, were there group
members who really weren't carrying their own weight? And then, at the
end of your assessment, estimate the percentage of the total amount of
work/effort done by each member. (Be sure your percentages sum to
100\%!)

For example, suppose you have 3 group members: X, Y and Z. In the
(unlikely) event that each member contributed equally, you could assign:

\begin{itemize}
\tightlist
\item
  33.3\% for member X, 33.3\% for member Y, and 33.3\% for member Z
\end{itemize}

Or in case person Z did twice as much work as each other member, you
could assign:

\begin{itemize}
\tightlist
\item
  25\% for member X, 25\% for member Y, and 50\% for member Z
\end{itemize}

Or if member Y didn't really do squat, you could assign:

\begin{itemize}
\tightlist
\item
  45\% for member X, 10\% for member Y, and 45\% for member Z
\end{itemize}

We'll find a fair way to synthesize the (possibly conflicting)
assessments within each group. And eventually we'll find a way to fairly
incorporate this assessment of effort and cooperation in each
individual's overall grade. Don't pressure one another to give everyone
glowing reports unless it's warranted, and don't feel pressured to share
your reports with one another. Just be fair to yourselves and to one
another. Let us know if you have any questions or if you run into any
problems.

\paragraph{Assessment Criteria}\label{assessment-criteria}

Your project will be evaluated based on the following criteria:

\begin{itemize}
\item
  General: Is the topic original, interesting, and substantial -- or is
  it trite, pedantic, and trivial? How much creativity, initiative, and
  ambition did the group demonstrate? Is the basic question driving the
  project worth investigating, or is it obviously answerable without a
  data-based study?
\item
  Design: Are the variables chosen appropriately and defined clearly,
  and is it clear how they were measured/observed? Can the effects of
  lurking variables be controlled for? Is there sufficient data to make
  meaningful conclusions?
\item
  Analysis: Are the chosen analyses appropriate for the
  variables/relationships under investigation, and are the assumptions
  underlying these analyses met? Do the analyses involve fitting and
  interpreting a multiple regression model? Are the analyses carried out
  correctly? Is there an effective mix of graphical, numerical, and
  inferential analyses? Did the group make appropriate conclusions from
  the analyses, and are these conclusions justified?
\item
  Technical Report: How effectively does the written report communicate
  the goals, procedures, and results of the study? Are the claims
  adequately supported? How well is the report structured and organized?
  Does the writing style enhance what the group is trying to
  communicate? How well is the report edited? Are the statistical claims
  justified? Are text and analyses effectively interwoven in the
  technical report? Clear writing, correct spelling, and good grammar
  are important.
\item
  Oral Presentation: How effectively does the oral presentation
  communicate the goals, procedures, and results of the study? Do the
  slides help to illustrate the points being made by the speaker without
  distracting the audience? Do the presenters seem to be well-rehearsed?
  Did they properly budget their time? Does she appear to be confident
  in what she is saying? Are her arguments persuasive?
\end{itemize}

\end{document}
